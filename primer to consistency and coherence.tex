\documentclass[UTF-8]{ctexbook}
\begin{document}
\title{内存一致性缓存一致性入门}
\author{Daniel J. Sorin, Mark D.Hill, David A.Wood 翻译:李默}
\date{2022.09.08}
\maketitle
\tableofcontents
\chapter*{摘要}
许多现代计算系统及多核处理器芯片在硬件上都支持共享内存。在有共享内存的系统中,不同的处理器核心可能对同一内存地址空间进行读写操作。对这样拥有共享内存的系统,它的内存一致性模型在架构上定义了它的内存组件能被观察到的行为的集合。一致性定义了读写操作的行为规则及它们如何与内存交互作用。为了实现所标称的一致性模型,一般这样的系统同时会使用特定的缓存一致性协议来保证多处缓存的数据时刻保持最新。 本入门手册的目的是为读者提供对内存一致性和缓存一致性模型的基本理解,包括技术问题与其解决方案。我们会在书中给出高度抽象的概念,同时也会给一些实际中的例子。
\\注: 内存一致性:  memory consistency,缓存一致性:  cache coherence. 二者的一致性的英文单词其实不同,所以其“一致”内涵实际有所不同。前者consistency字面意思是前后一致,即偏重时序上的行为特点,;而coherence 字面为协调,合理,所以更偏重在逻辑上的行为特点。但由于习惯用语,这里通用“一致性”。
\chapter{内存、缓存一致性简介}
许多现代计算系统及多核处理器芯片在硬件上都支持共享内存。在有共享内存的系统中,不同的处理器核心可能对同一内存地址空间进行读写操作。这种使用多核心、共享内存的系统设计一般是想追求高性能、低功耗、低成本等等特性。当然,如果没有了基本的正确性,那这些特性也将无价值。共享内存正确性乍看好像很简单,但实际上,正如本书将展示的,共享内存的正确性定义甚至都不是一件简单事,存在很多模糊点,更不用说实现一个“正确”的共享内存。而且这些问题和模棱两可需要在硬件实现过程弄清楚,因为修复一个硬件bug的成本是很高的。学术的人也应该弄清,明确对象,这样才能讨论某个提议的设计是否能正常工作。
\par 在实际研究中我们发现在研究内存正确性时从两维度出发将有助于问题的分析与解决:时序一致性(consistency)与逻辑一致性(coherence)。一个系统并不是必须要区分此二者,只是从我们的经验上看,此二者可以帮助我们将问题分解,各个击破。实际上,在共享内存实现过程中,普遍采用了这两个维度的分解方法。
\par consistency维度的目标是定义共享内存的正确性。consistency 的定义了读写如何作用到内存的规则。理想中consistency应该是一个简单的易于理解的模型。然而,定义共享内存的正确性比定义一个单核模型下的内存的正确性要模糊、朦胧地多。在单核模型下,内存正确性很好定义,这个正确性的行为可以与所有不正确性的行为有明确区分,这是因为在单核模型下,程序的输入具有明确的输出,即使这个cpu在内部执行时可能是乱序的。但对共享内存(即可能有多个处理器核心同时读写交互),它需要考虑来自多线程的并发读写,它可能会允许一些行为(即“正确”行为,是该内存模型下可能发生的行为),同时不允许另外一些行为(即“不正确”,是该内存模型承诺不会发生的行为)。之所以允许“一些”可能行为,而不是只允许“一种“行为,是因为在多线程并发读写的场景中无法明确这些线程孰前孰后,所以线程间可能有多种物理时序交错。多种正确可能性使得问题变得复杂,但为了在共享内存、多核系统上编写“正确”的程序,这些问题必须要得到解决。
\par 不同与consistency, 











\end{document}
